% % % % % % % % % % % % % % % % % % % % % % % % % % % % % % % % % % % % % % % % %
\section{Erweiterung um Typklassen und -konstruktoren}
% % % % % % % % % % % % % % % % % % % % % % % % % % % % % % % % % % % % % % % % %

\label{sec:erweiterung-typklassen}

In diesem Abschnitt soll erläutert werden, welche Änderungen an free-theorems vorgenommen werden müssen, um auch Typklassen und Typkonstruktoren miteinzubeziehen. In
Abschnitt 

usw.


% - - - - - - - - - - - - - - - - - - - - - - - - - - - - - - - - - - - - - - - - - - - - - - - - - - - - - - - - - - - - - - - - - - - - - - - - - - - - - -
\subsection{Erweiterungen der BasicSyntax}
% - - - - - - - - - - - - - - - - - - - - - - - - - - - - - - - - - - - - - - - - - - - - - - - - - - - - - - - - - - - - - - - - - - - - - - - - - - - - - -

Da Typvariablenapplikation in \cite{freetheorems} nicht vorgesehen ist, musste ein solches Konstrukt in die BasicSyntax-Struktur eingefügt werden. Hierbei ist zu beachten: Die Applikation
von Standardtypkonstruktoren, beispielsweise \texttt{[a]} für Listen, ist sehr wohl vorgesehen und wird in der BasicSyntax durch \texttt{TypeCon} im Datentyp \texttt{TypeExpression} dargestellt. \todo{Wahrscheinlich viel zu speziell} Diese Standardtypkonstruktoren werden in diesem Zusammenhang aber als Sonderfälle betrachtet \ref{sec:freie-theoreme}, um das System wirklich um eigene Typklassen zu erweitern, muss es auch möglich sein, Typkonstruktorvariablen auf Typen anzuwenden.

Die Erweiterung der BasicSyntax sieht hier so aus, dass ein neuer Konstruktor \texttt{TypeVarApp} zum Datentypen \texttt{TypeExpression} hinzugefügt wird. In Listing \ref{lst:typevarapp} ist ein Beispiel zu sehen.

\inputminted[tabsize=2]{haskell}{typevarapp.hs}

Abgesehen vom offensichtlichen Problem, dass dieser Konstruktor bei der Theoremgenerierung zu Fehlern beim Pattern Matching oder zumindest zu unerwünschtem Verhalten führt,
ist auch zu beachten, dass die in Abschnitt \ref{sec:check} erwähnte \texttt{check}-Funktion entsprechend angepasst wird.


% - - - - - - - - - - - - - - - - - - - - - - - - - - - - - - - - - - - - - - - - - - - - - - - - - - - - - - - - - - - - - - - - - - - - - - - - - - - - - -
\subsection{Erweiterung von Relation}
% - - - - - - - - - - - - - - - - - - - - - - - - - - - - - - - - - - - - - - - - - - - - - - - - - - - - - - - - - - - - - - - - - - - - - - - - - - - - - -

Da eine neue Relationskonstruktion eingeführt wird, muss auch der entsprechende Datentyp angepasst werden.


% - - - - - - - - - - - - - - - - - - - - - - - - - - - - - - - - - - - - - - - - - - - - - - - - - - - - - - - - - - - - - - - - - - - - - - - - - - - - - -
\subsection{Erweiterung von check}
% - - - - - - - - - - - - - - - - - - - - - - - - - - - - - - - - - - - - - - - - - - - - - - - - - - - - - - - - - - - - - - - - - - - - - - - - - - - - - -

Da in der ursprünglichen Version von free-theorems Typkonstruktorvariablen nicht gestattet waren, konnten einige
Fehlerüberprüfungen eingespart werden. Erlaubt man Typkonstruktorvariablen, muss man zusätzliche Fehlerfälle abdecken,
die auftreten können. Betrachten wir das folgende Beispiel:

\begin{minted}{haskell}
test :: Functor f => f a -> f
\end{minted}

Hierbei handelt es sich nicht um eine korrekte Haskell-Typsignatur: Die Variable f wird als Functor eingeführt und in der Signatur
einmal mit einem Parameter, einmal ohne Parameter aufgerufen. Selbst wenn man die Deklaration von Functor außer Acht lässt,
kann man deutlich sehen, dass eine der beiden Vorkommen von \texttt{f} fehlerhaft ist. \texttt{f} erwartet entweder einen
oder gar keinen Parameter, unterschiedliche Parameterzahlen sind nicht gestattet.

Da Functor bekannt ist, kann man einfach die Deklaration dieser Klasse betrachten:

\begin{minted}{haskell}
class Functor f where
   fmap :: (a -> b) -> f a -> f b
\end{minted}

Anhand der Deklaration der Klassenfunktionen kann man nun sehen, dass die Variable \texttt{f} stets auf einen einzelnen 
Parameter angewandt wird, das heißt, man kann bei allen Vorkommen der Klasse Functor davon ausgehen, dass diese
auf genau einen Parameter appliziert werden müssen. Natürlich kommt hier eine weitere Überprüfung ins Spiel: Es muss
sichergestellt werden, dass die Klassenvariable - in diesem Fall f - in sämtlichen Klassenfunktionen mit der gleichen Anzahl an
Parametern verwendet werden.

Die Überprüfung innerhalb einer Klasse auf gleiche Arität aller Vorkommen der Klassenvariable ist eine lokale Überprüfung -
es wird stets nur die Klassendeklaration benötigt, ein globaler Kontext muss nicht bekannt sein. Möchte man überprüfen,
ob Typkonstruktorvariablen in beliebigen Deklarationen die korrekte Anzahl an Parametern haben, dann muss man den
globalen Kontext betrachten: Die Deklaration der Klasse muss in Betracht gezogen werden.

Eine Besonderheit ist hierbei auch, dass eine Typvariable auf mehrere Klassen gleichzeitig eingeschränkt werden kann. So ist
das folgende Beispiel legitimer Haksell-Code:

\begin{minted}{haskell}
test :: (SomeClass1 f, SomeClass2 f) => f a -> f b
\end{minted}

Man muss also zusätzlich sicherstellen, dass alle Klassen, die für die gleichen Variablen angegeben sind, die gleiche Anzahl an
Parametern erwarten.
