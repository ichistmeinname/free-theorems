% % % % % % % % % % % % % % % % % % % % % % % % % % % % % % % % % % % % % % % % %
\section{Zusammenfassung}
% % % % % % % % % % % % % % % % % % % % % % % % % % % % % % % % % % % % % % % % %

In dieser Ausarbeitung wurde eine bereits existierende Haskell-Bibliothek zur automatischen Generierung freier Theoreme
beschrieben und eine Erweiterung dieser Bibliothek vorgestellt, mit der es möglich wird, freie Theoreme auch im Zusammenhang
mit Typkonstruktorklassen zu generieren.

In Kapitel \ref{sec:freie-theoreme} wurde vorgestellt, wie nach \cite{wadler} die Interpretation von Typen als Typrelationen
vonstatten geht und wie sich daraus wertvolle Theoreme ableiten lassen. Anhand dieser wurde dann in \ref{sec:free-theorems}
die Bibliothek \textit{free-theorems} \cite{freetheorems} erläutert und anhand einiger Beispieldaten veranschaulicht.

Da diese Bibliothek vorher nicht mit Typkonstruktorklassen umgehen konnte, lag der Fokus dieser Arbeit darauf, die Bibliothek dermaßen
zu erweitern, dass sie auch mit Typkonstruktorklassen arbeiten kann. Nebenbei wurden einige Kompatibilitätsprobleme beseitigt,
die ein Kompilieren der Bibliothek mit aktuellen Versionen des verwendeten GHC \cite{ghc} verhinderte.

Verschiedene Quellen stellen Möglichkeiten vor, freie Theoreme für Typsignaturen zu generieren, die Typkonstruktorklassen
verwenden. In dieser Arbeit findet die Vorgehensweise von Voigtländer \cite{voigtlander} Anwendung, die die relationale Interpretation
von Typsignaturen erweitert um eine Interpretation für Typkonstruktorvariablen-Abstraktion: Es wird über Funktionen
quantifiziert, die ihrerseits Typrelationen auf neue Relationen abbilden.

Mithilfe dieser Erweiterung lässt sich die Bibliothek jetzt auf einen breiteren Rahmen an Haskell-Programmen anwenden, da
Typsignaturen, die Typkonstruktorklassen verwenden, nicht länger wegfallen.